\documentclass[letterpaper,12pt]{article}
\usepackage{amsmath, amssymb, geometry}
\geometry{margin=1in}
\begin{document}

%INFO
\noindent
Christian DiPietrantonio \\
ME M311: Computational Methods to Viscious FLows \\
Video Assignment 01 \\
\today

\vspace{6pt}

\noindent\textbf{Problem Statement:} Watch all the video lectures, and derive the governing equation for the kinetic energy, defined as k = 1/2(u2 + v2 + w2). The technique involves multiplying the x-momentum equation by the u velocity, y-momentum by the v velocity, and z-momentum equation by the w velocity, and eventually add all three equations up. Show your derivation process.

\vspace{6pt}

\noindent\textbf{Solution:}

\vspace{6pt}

\noindent We must start this derivation with the momentum equations as mentioned above. First, we have the x-momentum equation:

\begin{equation}
\rho\frac{Du}{Dt} = -\frac{\partial p}{\partial x} + \frac{\partial \tau_{xx}}{\partial x} + \frac{\partial \tau_{yx}}{\partial y} + \frac{\partial \tau_{zx}}{\partial z} + S_{Mx}
\end{equation}

\noindent Next, we have the y-momentum equation:

\begin{equation}
\rho\frac{Dv}{Dt} = -\frac{\partial p}{\partial y} + \frac{\partial \tau_{xy}}{\partial x} + \frac{\partial \tau_{yy}}{\partial y}  + \frac{\partial \tau_{zy}}{\partial z} + S_{My}
\end{equation}

\noindent Finally, we have the z-momentum equation:

\begin{equation}
\rho\frac{Dw}{Dt} = -\frac{\partial p}{\partial z} + \frac{\partial \tau_{xz}}{\partial x} + \frac{\partial \tau_{yz}}{\partial y} + \frac{\partial \tau_{zz}}{\partial z} + S_{Mz}
\end{equation}

\noindent We can now multiply each equation by its respective velocity component, doing so yields the following for x-momentum:

\begin{equation}
\rho u \frac{Du}{Dt} = -u\frac{\partial p}{\partial x} + u\left(\frac{\partial \tau_{xx}}{\partial x} + \frac{\partial \tau_{yx}}{\partial y} + \frac{\partial \tau_{zx}}{\partial z}\right) + uS_{Mx}
\end{equation}

\noindent For y-momentum:

\begin{equation}
\rho v \frac{Dv}{Dt} = -v\frac{\partial p}{\partial y} + v\left(\frac{\partial \tau_{xy}}{\partial x} + \frac{\partial \tau_{yy}}{\partial y}  + \frac{\partial \tau_{zy}}{\partial z}\right) + vS_{My}
\end{equation}

\noindent And for z-momentum:

\begin{equation}
\rho w \frac{Dw}{Dt} = -w\frac{\partial p}{\partial z} + w\left(\frac{\partial \tau_{xz}}{\partial x} + \frac{\partial \tau_{yz}}{\partial y} + \frac{\partial \tau_{zz}}{\partial z}\right) + wS_{Mz}
\end{equation}

\noindent We can now add the three equations. Starting with the left hand side, we get:

\begin{equation}
    \rho \left(u\frac{Du}{Dt} + v\frac{Dv}{Dt} + w\frac{Dw}{Dt}\right) 
\end{equation}

\noindent Now, we have already defined kinetic energy, $k = \frac{1}{2}\left(u^2 + v^2 + w^2\right)$. If we take the material derrivative of k, we are left with:

\begin{equation}
\frac{Dk}{Dt} = \frac{D}{Dt}\left(\frac{1}{2}u^2\right) + \frac{D}{Dt}\left(\frac{1}{2}v^2\right) + \frac{D}{Dt}\left(\frac{1}{2}w^2\right)
\end{equation}

\noindent Applying the chain rule to each term leaves us with the following three equations:

\begin{equation}
\frac{D}{Dt}\left(\frac{1}{2}u^2\right) = u\frac{Du}{Dt}
\end{equation}

\begin{equation}
\frac{D}{Dt}\left(\frac{1}{2}v^2\right) = v\frac{Dv}{Dt}
\end{equation}

\begin{equation}
\frac{D}{Dt}\left(\frac{1}{2}w^2\right) = u\frac{Dw}{Dt}
\end{equation}

\noindent Therefore, we have:

\begin{equation}
\frac{Dk}{Dt} = u\frac{Du}{Dt} + v\frac{Dv}{Dt} + u\frac{Dw}{Dt}
\end{equation}

\noindent And we can re-write the left hand side of our equation as:

\begin{equation}
\rho\frac{Dk}{Dt}
\end{equation}

\noindent Now, we can turn our attention to the right hand side of our equation. Combining all of the pressure terms yeilds:

\begin{equation}
-u\frac{\partial p}{\partial x} - v\frac{\partial p}{\partial y} -w\frac{\partial p}{\partial z}
\end{equation}

\noindent This can also be expressed as:

\begin{equation}
-\left(u\frac{\partial p}{\partial x} + v\frac{\partial p}{\partial y} +w\frac{\partial p}{\partial z}\right) = -\vec{u} \cdot \nabla p
\end{equation}

\noindent We can now combine all of the viscious stress terms. Recalling that from x-momentum we have:

\begin{equation}
u\left(\frac{\partial \tau_{xx}}{\partial x} + \frac{\partial \tau_{yx}}{\partial y} + \frac{\partial \tau_{zx}}{\partial z}\right)
\end{equation}

\noindent From y-momentum we have:
\begin{equation}
v\left(\frac{\partial \tau_{xy}}{\partial x} + \frac{\partial \tau_{yy}}{\partial y}  + \frac{\partial \tau_{zy}}{\partial z}\right)
\end{equation}

\noindent And from z-momentum we have:
\begin{equation}
w\left(\frac{\partial \tau_{xz}}{\partial x} + \frac{\partial \tau_{yz}}{\partial y} + \frac{\partial \tau_{zz}}{\partial z}\right)
\end{equation}

\noindent Finally, we can combine the source terms to get:

\begin{equation}
uS_{Mx} + vS_{My} + wS_{Mz} 
\end{equation}

\noindent We can recognize this as the dot product of the velocity vector and the source term vector, therefore we can write:

\begin{equation}
uS_{Mx} + vS_{My} + wS_{Mz} = \vec{u} \cdot \vec{S_M}
\end{equation}

\noindent Finally, we can combine all aformentioned terms to get the governing equation for kinetic energy:

\begin{equation}
\boxed{
\begin{aligned}
\rho\frac{Dk}{Dt} ={}& -\vec{u} \cdot \nabla p \\[6pt]
&+ u\left(\frac{\partial \tau_{xx}}{\partial x} 
    + \frac{\partial \tau_{yx}}{\partial y} 
    + \frac{\partial \tau_{zx}}{\partial z}\right) \\[6pt]
&+ v\left(\frac{\partial \tau_{xy}}{\partial x} 
    + \frac{\partial \tau_{yy}}{\partial y}  
    + \frac{\partial \tau_{zy}}{\partial z}\right) \\[6pt]
&+ w\left(\frac{\partial \tau_{xz}}{\partial x} 
    + \frac{\partial \tau_{yz}}{\partial y} 
    + \frac{\partial \tau_{zz}}{\partial z}\right) \\[6pt]
&+ \vec{u} \cdot \vec{S_M}
\end{aligned}
}
\end{equation}

\end{document}
